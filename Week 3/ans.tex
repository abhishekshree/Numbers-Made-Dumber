\documentclass[12pt,oneside,reqno]{amsart}

\addtolength{\hoffset}{-2.25cm}
\addtolength{\textwidth}{4.5cm}
\addtolength{\voffset}{-2.5cm}
\addtolength{\textheight}{5cm}
\setlength{\parskip}{0pt}
\setlength{\parindent}{15pt}

\usepackage[dvipsnames]{xcolor}


\renewcommand{\baselinestretch}{1.5} 
\definecolor{gray}{HTML}{C7C8C9}

\usepackage{setspace}
\usepackage{amsthm}
\usepackage{amsmath}
\usepackage{amssymb}
\usepackage[colorlinks = true, linkcolor = black, citecolor = black, final]{hyperref}

\usepackage{graphicx}
\usepackage{multicol}
\usepackage{color}
\usepackage{ marvosym }
\usepackage{wasysym}
\newcommand{\ds}{\displaystyle}
\DeclareMathOperator{\sech}{sech}


\setlength{\parindent}{0in}

\pagestyle{empty}

\begin{document}

\thispagestyle{empty}
\begin{center}
    
    {\scshape \large  Assignment 3}
\end{center}
{\scshape Numbers Made Dumber} \hfill 
\hfill {\scshape Abhishek Shree}
\linebreak
{\scshape Project \#13}  \hfill {\scshape Roll: 200028}


\smallskip

\hrule

\bigskip
% First start
\paragraph*{1} Find the solution to the linear congruence 
\begin{gather*}
    x \equiv 3 \mod 5 \\
    x \equiv 4 \mod 11
\end{gather*}

\textbf{Sol.}
 $$N = 5*11 = 55
 \implies N_1 = 11, N_2 = 5$$
Therefore by Chinese Remainder Theorem, a particular solution would be,
$$ x_1 = 3*11*1 + 4*5*5 = 108$$
General solution would be,
$$x \equiv 108 \mod 55 \text{ which would be } \boxed{x \equiv 48 \mod 55}$$


\par\noindent\textcolor{gray}{\rule{\textwidth}{0.5pt}}
\smallskip
% First end

% Second start
\paragraph*{2.} For a positive integer p, define the positive integer n to be p-safe if n differs in absolutevalue by more than 2 from all multiples of p.  For example, the set of 10-safe numbers is $3, 4, 5, 6, 7, 13, 14, 15, 16, 17, 23...$ Find the number of positive integers less than or equal to 10,000 which are simultaneously 7-safe, 11-safe, and 13-safe.

\bigskip
\textbf{Sol.}
$x$ is 7-safe if, $x\equiv 3 \mod 7$ or $x\equiv 4 \mod 7$ i.e. 2 residues

Similarly, 11-safe numbers will have 6 residues (3, 4, 5, 6, 7, 8) and 13-safe numbers will have 8 residues (3, 4, 5, 6, 7, 8, 9, 10).

By Chinese Remainder Theorem, we will have a total of 96 residues (2*6*8) $\mod 1001$

Total numbers $\le10010$ satisfying this would be 960.

Removing 10006 and 10007 (values $ > 10000 $), we will have a total of \textbf{958} numbers.

\par\noindent\textcolor{gray}{\rule{\textwidth}{0.5pt}}
\smallskip
% Second end
\pagebreak
% Third start
\paragraph*{3.} Consider a number line consisting of all positive integers greater than 7.  A hole punch traverses the number line, starting from 7 and working its way up.  It checks each positive integer n and punches it if and only if $\binom{n}{7}$ is divisible by 12.  As the hole punch checks more and more numbers, the fraction of checked numbers that are punched approaches a limiting number $\rho$.  If $\rho$ can be written in the form $\frac{m}{n}$, where m and n are positive integers, find $m+n$.

\bigskip
\textbf{Sol.}
\par\noindent\textcolor{gray}{\rule{\textwidth}{0.5pt}}
\smallskip
% Third end

% Fourth start
\paragraph*{4.} Call a lattice point “visible” if the greatest common divisor of its coordinates is 1.  Prove that there exists a 100×100 square on the board none of whose points are visible.

\bigskip
\textbf{Sol.}
\par\noindent\textcolor{gray}{\rule{\textwidth}{0.5pt}}
\smallskip
% Fourth end

\end{document}